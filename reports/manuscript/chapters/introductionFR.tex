Tous les organismes subissent des mutations. Une mutation est une altération spontanée ou provoquée du support de l'information génétique. C'est un mécanisme qui joue un rôle clé dans l'histoire de la vie; en particulier, il explique l'apparition de nouveaux caractères héréditaires au sein d'une population. \\
    
Ces nouveaux caractères peuvent modifier la \emph{valeur sélective} d'un individu. La valeur sélective, appelée aussi \emph{fitness}, est un concept fondamental en biologie évolutive. Introduit par Darwin en 1859 dans son ouvrage \textit{On the Origin of Species} \cite{darwin1859origin}, ce terme désigne la capacité d'un individu ayant un certain génome à survivre et à se reproduire.\\
    
    On distingue trois types de mutations : 
        \begin{itemize}
            \item Certaines sont délétères voir létales, c'est-à-dire qu'elles réduisent le développement de l'individu ou le tuent.
            \item Certaines sont neutres et n'induisent aucun effet sur le développement. 
            \item Certaines sont bénéfiques et favorisent le développement de l'individu.
        \end{itemize}
            
    Les effets d'une mutation sur la valeur sélective d'un individu sont plus ou moins marqués, ce qui forme ainsi un continuum d'effets. \\
    
    La manière dont ces mutations affectent la valeur sélective d'un individu est une question centrale en biologie évolutive. En particulier, les biologistes s'intéressent à la densité de la répartition de ces effets \emph{(DFE, Density Fitness Effect)}.\\
    
    Connaître la forme de la DFE est une question essentielle. Dans un article publié en 2007 \cite{eyre2007distribution}, Eyre-Walker et al. donnent plusieurs exemples dans le but d'insister sur l'importance de son étude.
        \begin{itemize}
            \item Dans l'espèce humaine, le génome d'un individu possède plus d'une centaine de nouvelles mutations que l'on ne retrouve pas chez ses parents. L'étude de la DFE permet d'étudier leurs effets afin de savoir s'ils sont bénéfiques ou non.
            \item La DFE permet de comprendre et de quantifier la diversité génétique que l'on retrouve dans les maladies humaines ainsi que son évolution future.
            \item La DFE permet de prévoir les conséquences du maintien à petite taille d'une population d'animaux ou de plantes, comme dans les programmes d'élevage en captivité. \\
        \end{itemize}
        
    L'étude de la DFE est un passage obligé pour comprendre et prévoir la trajectoire évolutive d'une population et répondre à plusieurs questions qui restent encore en suspens.\\
    
    Cependant, il est difficile d'obtenir une estimation de la DFE \cite{bataillon2014effects}. En particulier, jusqu'à présent les moyens techniques ne permettaient pas d'observer précisément l'apparition d'une mutation dans le génome. \\
    
    %Pour se rendre compte qu'une mutation était apparue, il était nécessaire de le déduire à partir d'une variation du phénotype. \\


    Le point de départ de ce projet est un article de 2018 écrit par Lydia Robert et al. \cite{robert2018mutation}. Dans cet article, les auteur.ices présentent un nouveau protocole expérimental qui permet de suivre en temps réel l'apparition de nouvelles mutations chez \textit{E. coli}. \\
    
    Cette nouvelle méthode expérimentale permet d'obtenir de nouvelles données  afin d'estimer la DFE. En effet, il est possible grâce à ces données de savoir immédiatement quand une mutation apparaît. D'autres expériences permettent de regarder l'évolution de la valeur sélective d'une cellule au cours du temps. \\
    
    Dès lors, plusieurs méthodes sont possibles pour tenter d'estimation de la DFE à partir de ces observations expérimentales. Cependant il s'agit d'un problème inverse sévèrement mal posé. \\
    
    Dans ce mémoire, nous proposons d’étudier une question particulière d’estimation: 
    
        \begin{center}
            \textit{Inférer la DFE à partir de mesures expérimentales de la valeur sélective au cours du temps au sein d'une lignée cellulaire.}
        \end{center}
        
    Pour répondre à cette question, nous devrons prendre en compte que les données expérimentales sont extrêmement  bruitées. 
    Cette question rejoint une problématique plus large qui a de nombreuses autres applications possibles. \\



